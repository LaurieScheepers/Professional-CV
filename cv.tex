\documentclass[a4paper,10pt,notitlepage]{article}

%A Few Useful Packages
\usepackage{marvosym}
\usepackage{needspace}
\usepackage{textcomp}                       % extra symbols
\usepackage{fontspec} 					    % for loading fonts
\usepackage{xunicode,xltxtra,url,parskip} 	% other packages for formatting
\RequirePackage{color,graphicx}
\usepackage[usenames,dvipsnames]{xcolor}
%\usepackage[big]{layaureo} 				    % better formatting of the A4 page
                                            % an alternative to Layaureo can be ** \usepackage{fullpage} **
\usepackage{supertabular} 				    % for Grades
\usepackage{titlesec}					    % custom \section
\usepackage{wasysym}                        % some default emojis!
\usepackage{atbegshi}                       % http://ctan.org/pkg/atbegshi
\usepackage[none]{hyphenat}                 % No hyphenation
\usepackage[margin=0.8in]{geometry}
\AtBeginDocument{\AtBeginShipoutNext{\AtBeginShipoutDiscard}}

%Setup hyperref package, and colours for links
\usepackage{hyperref}
\definecolor{linkcolour}{rgb}{0,0.2,0.6}
\hypersetup{colorlinks,breaklinks,urlcolor=linkcolour, linkcolor=linkcolour}

%FONTS
\defaultfontfeatures{Mapping=tex-text}
%\setmainfont[SmallCapsFont = Fontin SmallCaps]{Fontin}
%%% modified for Karol Koziol for ShareLaTeX use
\setmainfont[
SmallCapsFont = Fontin-SmallCaps.otf,
BoldFont = Fontin-Bold.otf,
ItalicFont = Fontin-Italic.otf
]
{Fontin.otf}
%%%

%CV Sections inspired by: 
%http://stefano.italians.nl/archives/26
\titleformat{\section}{\Large\scshape\raggedright}{}{0em}{}[\titlerule]
%Tweak a bit off of the top margin
\addtolength{\voffset}{-1.0cm}

%Italian hyphenation for the word: ''corporations''
\hyphenation{im-pre-se}

\newenvironment{absolutelynopagebreak}
  {\par\nobreak\vfil\penalty0\vfilneg
   \vtop\bgroup}
  {\par\xdef\tpd{\the\prevdepth}\egroup
   \prevdepth=\tpd}

%--------------------BEGIN DOCUMENT----------------------
\begin{document}

\pagestyle{empty} % non-numbered pages

\font\fb=''[cmr10]'' %for use with \LaTeX command

%--------------------SECTIONS-----------------------------------
%Section: Personal Data

\addtolength{\voffset}{-0.8cm} % A bit of negative margin at top

\begin{absolutelynopagebreak}
	
	\begin{center}
		\vspace*{\stretch{0.8}}
		\begin{center}
			\Huge\textbf{Laurie Scheepers, Mr.}
		\end{center}
		\vspace*{\stretch{2.0}}
	\end{center}
		
	\section{Personal Data}
	
	\begin{tabular}{rl}
		\textsc{Place and Date of Birth:} & George, South Africa  | 21 Sep 1988                                 \\
		\textsc{Address:}                 & Oaklands, Unit nr 6, Schroder Rd, 7600, Stellenbosch, South Africa \\
		\textsc{Phone:}                   & +27 83 957 9726                                                     \\
		\textsc{ID:}                      & 8809215211083                                                       \\
		\textsc{email:}                   & \href{mailto:laurscheepers@gmail.com}{laurscheepers@gmail.com} (personal) \& \href{mailto:laurie@laurcode.com}{laurie@laurcode.com} (work)      \\
		\textsc{GitHub:}                  & \href{https://github.com/lauriescheepers}{https://github.com/lauriescheepers} \\
		\textsc{Website:}                 & \href{https://laurcode.com}{https://laurcode.com}
	\end{tabular}
	
	%Section: Work Experience at the top
	
	\section{Work Experience}
	\begin{tabular}{r|p{11cm}}
	
	    \textsc{May 2018 - Present}  & Service Provider at \href{www.nomanini.com}{Nomanini}
		\\&\emph{Senior Android Developer}\\&\footnotesize{Freelance, contract-based work for development of Nomanini's Android apps: \href{https://play.google.com/store/apps/details?id=zm.co.broadpay.android}{Broadpay}, Moza D'Agente, and Tikambe (PanAfrica).}\\
		
		\multicolumn{2}{c}{} \\
	
    	\textsc{Feb 2018 - Jul 2018}  & Subcontractor at \href{www.uber5.com}{Uber5}
		\\&\emph{Freelance \& Contract-based development work}\\&\footnotesize{Freelance Android development for Uber5 as a subcontractor for a client of theirs.}\\
		
		\multicolumn{2}{c}{} \\
		
		\textsc{Nov 2017 - Forever}  & Owner at \href{www.laurcode.com}{LaurCode}
		\\&\emph{Freelance \& Contract-based development work}\\&\footnotesize{Decided to take my career to the next step by making my own website and starting to promote myself as an independent freelancer, specializing in mobile development \& design. A detailed history of projects I have been involved in is also available on the site. Go visit \href{https://www.laurcode.com}{laurcode.com} and check it out!\\
		
		\multicolumn{2}{c}{} \\
		
		\textsc{Mar 2017 - Ongoing}  & Developer at \href{www.sudonum.com}{Sudonum}
		\\&\emph{Click-To-Call Library}\\&\footnotesize{Private contracting freelance work at Sudonum, a call tracking and communication company, responsible for developing a VoIP library for Android \& Web that uses \href{https://en.wikipedia.org/wiki/WebRTC}{WebRTC} as a communication protocol. There is still ongoing work done scheduled for 2018 when the API's are ready to be consumed by a mobile app.} \\
		
		\multicolumn{2}{c}{} \\
		\textsc{Sep 2016 - Jan 2018}  & Senior Android Developer at \href{www.kagisomedia.co.za}{Kagiso Media}                                               \\&\emph{Realtime Audio Streaming Apps \& LiveAmp}\\&\footnotesize{Freelance Android developer responsible for releasing new versions of the apps, \href{https://play.google.com/store/apps/details?id=com.kagiso.jacarandafm}{Jacaranda FM},  \href{https://play.google.com/store/apps/details?id=com.kagiso.ecr}{East Coast Radio}, and \href{https://play.google.com/store/apps/details?id=com.kagiso.soundbar}{Soundbar}. I also developed an app for a popular local TV show, called  \href{https://play.google.com/store/apps/details?id=com.kagiso.liveamp}{LiveAmp}. I still occasionally help with development or questions about all these apps for Kagiso. The apps can be viewed (with ratings minimum 4.1 and higher) \href{https://play.google.com/store/apps/developer?id=Kagiso+Media}{here}}. \\
		
		\multicolumn{2}{c}{} \\
		\textsc{Aug 2015 - Sep 2016} & Head of Product Development at \href{www.thereachtrust.org}{The Reach Trust} \\                                                                  &\emph{LevelUp -  Educational Android app} \\
		                            &\footnotesize{General Head of Dev \& Lead of the Android team creating the  flag-ship product, \textbf{LevelUp}, available on the \href{https://play.google.com/store/apps/details?id=org.mylevelup}{Play Store}. The Reach Trust is a public benefit organization that became their own business after \href{https://memeburn.com/2015/10/mxit-confirms-its-shutting-up-shop/}{Mxit's end}, mainly focusing on \href{https://it-online.co.za/2015/10/26/reach-trust-will-use-mxit-for-education/}{education and improving people's lives}. All of The Reach Trust apps (together with their ratings) can be found on the \href{https://play.google.com/store/apps/dev?id=7356513661681471434}{Play Store}.} \\
		 
		\multicolumn{2}{c}{} \\
		\textsc{Jan 2014 - Aug 2015} & Senior Android Developer at \href{www.mxit.com}{Mxit} \\
		                             & \emph{Team Lead of Mxit Android Client} \\
		                             & \emph{Team Lead of Broadcasts POC - One-to-many communication app} \\
		                             & \footnotesize{Full-time work as team lead, managing the \href{https://en.wikipedia.org/wiki/Mxit}{Mxit} developers of the Android client (average rating of 4.6 while on the Play Store), while maintaining the J2ME and Blackberry apps. During this time we focused on implementing \textsc{VoIP} functionality and a new feature called \textit{Make Friends}. I continued working for Mxit until the very end, when the servers were switched off.} \\
		 
		\multicolumn{2}{c}{} \\
		\textsc{Jan 2012 - Jan 2014} & Developer at \href{www.mxit.com}{Mxit} \\
		                             & \emph{Mxit - Social networking and instant messaging app} \\
		                             & \footnotesize{Started my career as developer on the J2ME version of Mxit, used by millions of people around the world. Climbed the rungs of the ladder, later becoming team lead of J2ME and Blackberry teams and also finally joining the Android team (circa 2013), which I was very excited about. For history regarding the rise \& fall of Mxit, read \href{https://en.wikipedia.org/wiki/Mxit}{here} and \href{https://www.moneyweb.co.za/news/companies-and-deals/how-did-mxit-go-so-wrong/}{here}.} \\
		
	\end{tabular}
	
\end{absolutelynopagebreak}

\begin{absolutelynopagebreak}
	
	\section{Courses}
	\begin{tabular}{r|p{11cm}}
	
		\textsc{2018} &                                                                                                                  
		\href{https://www.udemy.com/ios-11-app-development-bootcamp/learn/v4/overview}{iOS 11 \& Swift 4 - The Complete iOS App Development Bootcamp}\\&\footnotesize{A planned iOS course (already enrolled) that I will complete in 2018. Decided to focus on Swift rather than Objective C.} \\
		
		\multicolumn{2}{c}{} \\
		\textsc{2018} &                                                                                                                  
		\href{https://www.udemy.com/the-complete-react-native-and-redux-course/learn/v4/overview}{The Complete React Native and Redux Course}\\&\footnotesize{Planned React Native course (already enrolled) that I will complete in 2018. React is the only cross-platform technology that I want to learn as the others are not as advanced.} \\
		
		\multicolumn{2}{c}{} \\
		\textsc{2017} &                                                                                                                  
		\href{https://www.udemy.com/app-design-with-sketch-ui-and-ux/}{App Design with Sketch: UI and UX}\\&\footnotesize{A design course about Sketch that I am currently following to upskill my design work.} \\
		
		\multicolumn{2}{c}{} \\
		\textsc{2017} &                                                                                                                  
		\href{https://www.udemy.com/aftereffects/}{After Effects CS6: The Complete Guide to Adobe After Effects}\\&\footnotesize{A design course about Adobe After Effects that I am currently following to upskill myself in creating more advanced animations.} \\
		
		\multicolumn{2}{c}{} \\
		\textsc{2016} & \href{https://www.udemy.com/unitycourse/learn/v4/}{Learn to Code by Making Games - Complete C\# Unity Developer} \\&\footnotesize{Course in Unity, for developing cross-platform games.} \\
		
	\end{tabular}
	
	\section{Personal Projects}
	\begin{tabular}{r|p{11cm}}
	
	    \textsc{2018} &                                                                                                                  
		Retire\\&\footnotesize{A financial Android app (work in progress), simply called Retire. This "robo-advice" app only has a few financial inputs required and then will show a nice output of your money accumulation phase and the drawdown phase of all those savings over the years that you can use at your leisure in your retirement. Basically it's going to be a simple, user-friendly retirement-planning app available for anyone. \textregistered LaurCode \\
		
		\multicolumn{2}{c}{} \\
		
		\textsc{2017} &                                                                                                                  
		\href{www.laurcode.com}{www.laurcode.com}\\&\footnotesize{The website I created to represent my freelance developer business, LaurCode. It contains some other interesting things as well, like album reviews, and I also publish a blog article once in a blue moon.} \\
		
		\multicolumn{2}{c}{} \\
		\textsc{2015} &                                                                                                                  
		\href{https://play.google.com/store/apps/details?id=za.co.inflationcalc}{SA Inflation Calculator}\\&\footnotesize{A simple Android app that calculates what historical, inflation adjusted money is worth today in South Africa.} \\
		
		\multicolumn{2}{c}{} \\
		
	\end{tabular}
	
	\\\\
	
    \footnotesize{I have a whole Google Drive folder full of future apps I want to make but that is too exhaustive to include in a CV}.
	
	%Section: Education
	\section{Education}
	\begin{tabular}{r|p{11cm}}
		
		\textsc{2007-2011} & \textbf{Bachelors of Engineering: \textsc{Electrical, Electronic Engineering \& Computer Science}}, called \href{https://en.wikipedia.org/wiki/Computer_engineering}{Computer Engineering} elsewhere. \\
		& \href{https://en.wikipedia.org/wiki/Stellenbosch_University}{Stellenbosch University}, Stellenbosch, South Africa \\
		
		\multicolumn{2}{c}{}\\
		
		& \textsc{Thesis}: ``Automatic Classification of Ethnicity based on South African Names'', a classification program done in C++ using probabilistic mathematical models and a subset of \textbf{Machine Learning} called \textbf{Supervised Learning}. \small Final Grade Achieved: 75\% \textit{Cum Laude} \\ \\
		&\normalsize \textsc{Overall Average During Studies}: 69\% \\
		
		\multicolumn{2}{c}{}\\
		
		\textsc{2002-2006} & \href{https://en.wikipedia.org/wiki/Paarl_Boys_27_High_School}{Paarl Boys High} \\ & Paarl, South Africa \\ \\
		&\textsc{Subjects}: Mathematics, Physical Science, Afrikaans, English, Latin, Accounting \\ \\
		&\normalsize \textsc{Matric Results}: 2nd Dux, 6 A's, 91\% average, subject results available on request. \\
		
	\end{tabular}
	
\end{absolutelynopagebreak}

\begin{absolutelynopagebreak}

	%Section: Scholarships and additional info
\section{Honours and Rewards}
	\begin{tabular}{r|p{11cm}}
		\textsc{2002-2006}  & Paarl Boys High: Top 10 academically each year, awarded full academic honours in Grade 11 and awarded 2nd Dux Medal in Matric. \\
		\textsc{2006}       & James Grace Eisteddfod Cup: Best classical guitarist.                                                                          \\
		\textsc{2006}       & Matric results: Achieved 3rd place in Western Cape in Latin.                                                                   \\
		\textsc{2006}       & Placed 14th nationwide in SA Taalbond Bilingualism Examination.                                                                \\
		\textsc{2006}       & Placed 78th nationwide in De Beers English Olympiad.                                                                           \\
		\textsc{2008, 2009} & Invited to join Golden Key International Honour Society (top 15\% of undergraduate students).                                  
	\end{tabular}

%Section: Languages

\addtolength{\voffset}{-0.4cm} % A bit of extra negative margin at top
	
	\section{Languages}
	\begin{tabular}{r|p{11cm}}
		\textsc{Afrikaans} & Primary Language       \\
		\textsc{English}   & Fluent                 \\
		\textsc{Latin}     & Studied in High School \\
	\end{tabular}
	
	\section{Digital Skills}
	
	\textit{Feel free to ask more details about my skills in development and tools used etc. The following is just a structured summary.} \\
	
	\footnotesize{In general I believe that a programming language is just a language to learn, get used to the syntax and gain experience in - all languages differ in their feature sets and also their purposes, one is not more difficult or \textit{better} than the other, they are just different in their approaches to general Computer Science principles.} \\
	
	\begin{tabular}{r|p{11cm}}
		
		\textsc{Programming Languages}    & \textbf{Advanced}: \textsc{Java 1.3 - 1.8 (Mobile: Android, Blackberry, J2ME)}, \textsc{C}, \textsc{C++} \\
		                                  & \emph{Intermediate}: \textsc{C\#}, \textsc{HTML}, \textsc{JavaScript}, \textsc{CSS}, \textsc{Python}, Excel, VB, MarkDown, {\fb \LaTeX} \setmainfont[SmallCapsFont=Fontin-SmallCaps.otf]{Fontin.otf} \\
		                                  & \emph{Starting to upskill myself in}: \textsc{Swift}, \textsc{Objective-C}, \textsc{React Native} \\
		                                  
		\multicolumn{2}{c}{}\\
		
		\textsc{Design}                   & \textsc{Learning more as I go along}: \textsc{Adobe Photoshop}, \textsc{Adobe Illustrator}, \textsc{Adobe After Effects}, \textsc{Sketch}, \textsc{Paint.NET}, Various Online Tools (AppLaunchPad, Android Asset Generator, FlatIcon, SVG things, etc\ldots) \\
		
		\multicolumn{2}{c}{}\\
		
		\textsc{Operating Systems}        & \textsc{MacOS} (main daily driver), \textsc{Linux} (Ubuntu, Linux Mint), \textsc{Windows} (from 3.1), \textsc{MS-DOS} (when young) \\
		 
		\multicolumn{2}{c}{} \\
		 
		\textsc{Development Environments} & \textsc{Android Studio}, \textsc{IntelliJ}, \textsc{WebStorm}, \textsc{Eclipse}, \textsc{Visual Studio}, \textsc{XCode}, \textsc{Adobe Creative Cloud Products} \\
		 
		\multicolumn{2}{c}{} \\
		 
		\textsc{Version Control Systems}  & \textsc{Git} (GitHub, GitLab), \textsc{SVN}, Own versioning used on cloud storage for other things: \textsc{Google Drive}, \textsc{DropBox}, \textsc{Confluence}, \textsc{iCloud} \\    
		 
	\end{tabular}
	
\end{absolutelynopagebreak}
	
	%\newpage
	%\hypertarget{gmat}{\textsc{Gmat}\setmainfont{LMRoman10 Regular}\textregistered\setmainfont[SmallCapsFont=Fontin-SmallCaps]{Fontin-Regular}}
	
	%\XeTeXpdffile ''GMAT.pdf'' page 1 scaled 800

\end{document}
