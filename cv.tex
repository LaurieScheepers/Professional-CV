\documentclass[a4paper,10pt,notitlepage]{article}

%A Few Useful Packages
\usepackage{marvosym}
\usepackage{needspace}
\usepackage{fontspec} 					    % for loading fonts
\usepackage{xunicode,xltxtra,url,parskip} 	% other packages for formatting
\RequirePackage{color,graphicx}
\usepackage[usenames,dvipsnames]{xcolor}
%\usepackage[big]{layaureo} 				    % better formatting of the A4 page
                                            % an alternative to Layaureo can be ** \usepackage{fullpage} **
\usepackage{supertabular} 				    % for Grades
\usepackage{titlesec}					    % custom \section
\usepackage{wasysym}                        % some default emojis!
\usepackage{atbegshi}                       % http://ctan.org/pkg/atbegshi
\usepackage[none]{hyphenat}                 % No hyphenation
\usepackage[margin=0.8in]{geometry}
\AtBeginDocument{\AtBeginShipoutNext{\AtBeginShipoutDiscard}}

%Setup hyperref package, and colours for links
\usepackage{hyperref}
\definecolor{linkcolour}{rgb}{0,0.2,0.6}
\hypersetup{colorlinks,breaklinks,urlcolor=linkcolour, linkcolor=linkcolour}

%FONTS
\defaultfontfeatures{Mapping=tex-text}
%\setmainfont[SmallCapsFont = Fontin SmallCaps]{Fontin}
%%% modified for Karol Koziol for ShareLaTeX use
\setmainfont[
SmallCapsFont = Fontin-SmallCaps.otf,
BoldFont = Fontin-Bold.otf,
ItalicFont = Fontin-Italic.otf
]
{Fontin.otf}
%%%

%CV Sections inspired by: 
%http://stefano.italians.nl/archives/26
\titleformat{\section}{\Large\scshape\raggedright}{}{0em}{}[\titlerule]
%Tweak a bit off of the top margin
\addtolength{\voffset}{-1.0cm}

%Italian hyphenation for the word: ''corporations''
\hyphenation{im-pre-se}

\newenvironment{absolutelynopagebreak}
  {\par\nobreak\vfil\penalty0\vfilneg
   \vtop\bgroup}
  {\par\xdef\tpd{\the\prevdepth}\egroup
   \prevdepth=\tpd}

%--------------------BEGIN DOCUMENT----------------------
\begin{document}

\pagestyle{empty} % non-numbered pages

\font\fb=''[cmr10]'' %for use with \LaTeX command

%--------------------SECTIONS-----------------------------------
%Section: Personal Data

\addtolength{\voffset}{-0.8cm} % A bit of negative margin at top

\begin{absolutelynopagebreak}
	
	\begin{center}
		\vspace*{\stretch{0.8}}
		\begin{center}
			\Huge\textbf{Laurie Scheepers, Mr.}
		\end{center}
		\vspace*{\stretch{2.0}}
	\end{center}
		
	\section{Personal Data}
	
	\begin{tabular}{rl}
		\textsc{Place and Date of Birth:} & George, South Africa  | 21 Sep 1988                                 \\
		\textsc{Address:}                 & Oaklands, Unit nr 6, Schroder Rd, 7600, Stellenbosch, South Africa \\
		\textsc{Phone:}                   & +27 83 957 9726                                                     \\
		\textsc{ID:}                      & 8809215211083                                                       \\
		\textsc{email:}                   & \href{mailto:laurscheepers@gmail.com}{laurscheepers@gmail.com} (personal) \& \href{mailto:laurie@laurcode.com}{laurie@laurcode.com} (work)      \\
		\textsc{GitHub:}                  & \href{https://github.com/lauriescheepers}{https://github.com/lauriescheepers}
	\end{tabular}
	
	%Section: Work Experience at the top
	
	\section{Work Experience}
	\begin{tabular}{r|p{11cm}}
		
		\textsc{Nov 2017 - Forever}  & Owner at \href{www.laurcode.com}{LaurCode}
		\\&\emph{Freelance \& Contract development work}\\&\footnotesize{Decided to take my career to the next step by registering my own domain and starting to promote myself as an independent freelancer, specializing in mobile development \& design. Website is still in development, but I can be contacted for work at \href{mailto:laurie@laurcode.com}{laurie@laurcode.com}.} \\
		
		\multicolumn{2}{c}{} \\
		
		\textsc{Mar 2017 - Ongoing}  & Developer at \href{www.sudonum.com}{Sudonum}
		\\&\emph{Click-To-Call Library}\\&\footnotesize{Private contracting freelance work at Sudonum, a call tracking and communication company, responsible for developing a VoIP library for Android \& Web that uses \href{https://en.wikipedia.org/wiki/WebRTC}{WebRTC} as a communication protocol. There is still ongoing work done scheduled for later in the year when the API's are ready to be consumed by a mobile app. WebRTC is very powerful and interesting, and I think there will be a LOT more WebRTC tools in the future as it grows and becomes part of
		\href{https://en.wikipedia.org/wiki/Internet_of_things}{IoT}.} \\
		
		\multicolumn{2}{c}{} \\
		\textsc{Sep 2016 - Ongoing}  & Senior Android Developer at \href{www.kagisomedia.co.za}{Kagiso Media}                                               \\&\emph{Realtime Audio Streaming Apps \& LiveAmp}\\&\footnotesize{Sole Android developer responsible for releasing new versions of the apps, \href{https://play.google.com/store/apps/details?id=com.kagiso.jacarandafm}{Jacaranda FM},  \href{https://play.google.com/store/apps/details?id=com.kagiso.ecr}{East Coast Radio}, and \href{https://play.google.com/store/apps/details?id=com.kagiso.soundbar}{Soundbar}. I also developed an app for a popular local TV show, called  \href{https://play.google.com/store/apps/details?id=com.kagiso.liveamp}{LiveAmp}. I am still doing private freelance work on a contract basis on all these apps for Kagiso. They can all be viewed (with all their ratings, which range from 4.2 - 4.8) \href{https://play.google.com/store/apps/developer?id=Kagiso+Media}{here}}. \\
		
		\multicolumn{2}{c}{} \\
		\textsc{Aug 2015 - Sep 2016} & Head of Product Development at \href{www.thereachtrust.org}{The Reach Trust} \\                                                                  &\emph{LevelUp -  Educational Android app} \\
		                            &\footnotesize{General Head of Dev \& Lead of the Android team creating the  flag-ship product, \textbf{LevelUp}, available on the \href{https://play.google.com/store/apps/details?id=org.mylevelup}{Play Store}, it currently has a rating of 4.4. The Reach Trust is a public benefit organization that became their own business after \href{https://memeburn.com/2015/10/mxit-confirms-its-shutting-up-shop/}{Mxit's end}, mainly focusing on \href{https://it-online.co.za/2015/10/26/reach-trust-will-use-mxit-for-education/}{education and improving people's lives}. The company, after some financial difficulty and retrenchments, have since however gone on to make amazing new apps focusing on early-childhood-development (ECD) all of which (together with their ratings) can be found on the \href{https://play.google.com/store/apps/dev?id=7356513661681471434}{Play Store}} \\
		 
		\multicolumn{2}{c}{} \\
		\textsc{Jan 2014 - Aug 2015} & Senior Android Developer at \href{www.mxit.com}{Mxit} \\
		                             & \emph{Team Lead of Mxit Android Client} \\
		                             & \emph{Team Lead of Broadcasts - One-to-many communication app} \\
		                             & \footnotesize{Full-time work as team lead, managing the \href{https://en.wikipedia.org/wiki/Mxit}{Mxit} developers of the Android client (average rating of 4.6 while on the Play Store), while maintaining the J2ME and Blackberry apps. During this time we focused on implementing \textsc{VoIP} functionality and a new feature called \textit{Make Friends}. Also later became the lead developer responsible for creating an Android app called \textsc{Broadcasts} (unfortunately never released), an app meant to make it easy for users to receive important info via push notifications from subscribed ``Channels''.} \\
		 
		\multicolumn{2}{c}{} \\
		\textsc{Jan 2012 - Jan 2014} & Developer at \href{www.mxit.com}{Mxit} \\
		                             & \emph{Mxit - Social networking and instant messaging app} \\
		                             & \footnotesize{Started my career as developer on the J2ME version of Mxit, used by millions of people around the world. Climbed the rungs of the ladder, later becoming team lead of J2ME and Blackberry teams and also finally joining the Android team (circa 2013), which I was very excited about. For history regarding the rise \& fall of Mxit, read \href{https://en.wikipedia.org/wiki/Mxit}{here} and \href{https://www.moneyweb.co.za/news/companies-and-deals/how-did-mxit-go-so-wrong/}{here}.} \\
		
	\end{tabular}
	
\end{absolutelynopagebreak}

\begin{absolutelynopagebreak}
	
	\section{Courses \& Personal Projects}
	\begin{tabular}{r|p{11cm}}
	
		\textsc{Planned for Dec 2017} &                                                                                                                  
		\href{https://www.udemy.com/ios-11-app-development-bootcamp/learn/v4/overview}{iOS 11 \& Swift 4 - The Complete iOS App Development Bootcamp}\\&\footnotesize{A planned iOS course (already enrolled) that I will complete in 2018. Decided to focus on Swift rather than Objective C (that can be learnt on the side - it's just a language).} \\
		
		\multicolumn{2}{c}{} \\
		\textsc{Planned for Dec 2017} &                                                                                                                  
		\href{https://www.udemy.com/the-complete-react-native-and-redux-course/learn/v4/overview}{The Complete React Native and Redux Course}\\&\footnotesize{Planned React Native course (already enrolled) that I will complete in 2018. React is the only cross-platform technology that I am willing to learn as the others are not good enough in my opinion.} \\
		
		\multicolumn{2}{c}{} \\
		\textsc{Oct 2017 - Present} &                                                                                                                  
		\href{https://www.udemy.com/app-design-with-sketch-ui-and-ux/}{App Design with Sketch: UI and UX}\\&\footnotesize{A design course about Sketch that I am currently following to upskill my passions about design.} \\
		
		\multicolumn{2}{c}{} \\
		\textsc{Oct 2017 - Present} &                                                                                                                  
		\href{https://www.udemy.com/aftereffects/}{After Effects CS6: The Complete Guide to Adobe After Effects}\\&\footnotesize{A design course about Adobe After Effects that I am currently following to upskill my passions about animations.} \\
		
		\multicolumn{2}{c}{} \\
		\textsc{Oct 2016 - Present} & \href{https://www.udemy.com/unitycourse/learn/v4/}{Learn to Code by Making Games - Complete C\# Unity Developer} \\&\footnotesize{Ongoing course in Unity, for developing cross-platform games.} \\
		
		\multicolumn{2}{c}{} \\
		\textsc{Nov 2017 - Present} &                                                                                                                  
		\href{www.laurcode.com}{www.laurcode.com}\\&\footnotesize{My own personal website that will contain my personal development portfolio, details about my business plan as freelancer/contractor, as well as info on my thoughts about programming and Android specifically. It will also contain personal musings about interesting things in life. It's just basically a place to put down all my ideas. Website coming soon.} \\
		
		\multicolumn{2}{c}{} \\
		\textsc{Sep 2015 - Present} &                                                                                                                  
		\href{https://play.google.com/store/apps/details?id=za.co.inflationcalc}{SA Inflation Calculator}\\&\footnotesize{A very simple app that calculates what historical, inflation adjusted money is worth today in South Africa.} \\
		
		\multicolumn{2}{c}{} \\
		\textsc{PLANNED}            &                                                                                                                  
		\href{www.laurcode.com}{DiceApp}\\&\footnotesize{Inspired by Luke Rhinehart’s novel \href{https://en.wikipedia.org/wiki/The_Dice_Man}{The Dice Man}, it tells the story of a psychiatrist who begins making life decisions based on the casting of dice. I am planning on making an interesting, interactive, voice-driven app about the concepts and philosophies explained in the book. More details on request. All Rights Reserved.} \\
		
		\multicolumn{2}{c}{} \\
		\textsc{PLANNED}            &                                                                                                                  
		\href{www.laurcode.com}{RadioRewind}\\&\footnotesize{An app idea my father and I had that I am planning on implementing as soon as possible. Basic description: an app containing features, e.g. pausing, rewinding, recording, of a PVR/DVR (as used with DSTV), but just for radio and live streamed audio played on your computer/mobile device. I may sell this idea and implement it for a company I am contracting for, I still have to decide. More details on request. All Rights Reserved.} \\
		
	\end{tabular}
	
	%Section: Education
	\section{Education}
	\begin{tabular}{r|p{11cm}}
		
		\textsc{2007-2011} & \textbf{Bachelors of Engineering: \textsc{Electrical, Electronic Engineering \& Computer Science}}, also called \href{https://en.wikipedia.org/wiki/Computer_engineering}{Computer Engineering} elsewhere. \\
		& \href{https://en.wikipedia.org/wiki/Stellenbosch_University}{Stellenbosch University}, Stellenbosch, South Africa \\
		
		\multicolumn{2}{c}{}\\
		
		& \textsc{Thesis}: ``Automatic Classification of Ethnicity based on South African Names'', a classification program done in C++ using probabilistic mathematical models and a subset of \textbf{Machine Learning} called \textbf{Supervised Learning}. \small Final Grade Achieved: 75\% \textit{Cum Laude} \\
		&\normalsize \textsc{Overall Average During Studies}: ~69\% \\
		
		\multicolumn{2}{c}{}\\
		
		\textsc{2002-2006} & \href{https://en.wikipedia.org/wiki/Paarl_Boys_27_High_School}{Paarl Boys High} \\ & \normalsize Paarl, South Africa \\
		&\textsc{Subjects}: Mathematics, Physical Science, Afrikaans, English, Latin, Accounting \\
		&\normalsize \textsc{Matric Results}: 2nd Dux, 6 A's, 91\% average, subject results on request. \\
		
	\end{tabular}
	
\end{absolutelynopagebreak}

\begin{absolutelynopagebreak}

	%Section: Scholarships and additional info
\section{Honours and Rewards}
	\begin{tabular}{r|p{11cm}}
		\textsc{2002-2006}  & Paarl Boys High: Top 10 academically each year, awarded full academic honours in Grade 11 and awarded 2nd Dux Medal in Matric. \\
		\textsc{2006}       & James Grace Eisteddfod Cup: Best classical guitarist.                                                                          \\
		\textsc{2006}       & Matric results: Achieved 3rd place in Western Cape in Latin.                                                                   \\
		\textsc{2006}       & Placed 14th nationwide in SA Taalbond Bilingualism Examination.                                                                \\
		\textsc{2006}       & Placed 78th nationwide in De Beers English Olympiad.                                                                           \\
		\textsc{2008, 2009} & Invited to join Golden Key International Honour Society (top 15\% of undergraduate students).                                  
	\end{tabular}

%Section: Languages

\addtolength{\voffset}{-0.4cm} % A bit of extra negative margin at top
	
	\section{Languages}
	\begin{tabular}{r|p{11cm}}
		\textsc{Afrikaans} & Primary Language       \\
		\textsc{English}   & Fluent                 \\
		\textsc{Latin}     & Studied in High School \\
	\end{tabular}
	
	\section{Digital Skills}
	
	\textit{Feel free to ask more details about my skills in development and tools used etc. The following is just a structured summary.} \\
	
	\footnotesize{In general I believe that a programming language is just a language to learn, get used to the syntax and gain experience in - all languages differ in their feature sets and also their purposes, one is not more difficult or \textit{better} than the other, except for esoteric things like \href{https://en.wikipedia.org/wiki/Brainfuck}{BrainFuck} (which is made difficult on purpose as an exercise to the implementer and reader).} \\
	
	\begin{tabular}{r|p{11cm}}
		
		\textsc{Programming Languages}    & \textbf{Advanced}: \textsc{Java 1.3 - 1.8 (Mobile: Android, Blackberry, J2ME)}, \textsc{C}, \textsc{C++} \\
		                                  & \emph{Intermediate}: \textsc{C\#}, \textsc{HTML}, \textsc{JavaScript}, \textsc{CSS}, \textsc{Python}, Excel, VB, MarkDown, {\fb \LaTeX} \setmainfont[SmallCapsFont=Fontin-SmallCaps.otf]{Fontin.otf} \\
		                                  & \emph{Starting to upskill myself in}: \textsc{Swift}, \textsc{Objective-C}, \textsc{React Native} \\
		                                  
		\multicolumn{2}{c}{}\\
		
		\textsc{Design}                   & \textsc{Learning more as I go along}: \textsc{Adobe Photoshop}, \textsc{Adobe Illustrator}, \textsc{Adobe After Effects}, \textsc{Sketch}, \textsc{Paint.NET}, Various Online Tools (AppLaunchPad, Android Asset Generator, FlatIcon, SVG things, etc\ldots) \\
		
		\multicolumn{2}{c}{}\\
		
		\textsc{Operating Systems}        & \textsc{MacOS} (main daily driver), \textsc{Linux} (Ubuntu, Linux Mint), \textsc{Windows} (from 3.1), \textsc{MS-DOS} (when young) \\
		 
		\multicolumn{2}{c}{} \\
		 
		\textsc{Development Environments} & \textsc{Android Studio}, \textsc{IntelliJ}, \textsc{WebStorm}, \textsc{Eclipse}, \textsc{Visual Studio}, \textsc{XCode} \textsc{Adobe Creative Cloud Products} \\
		 
		\multicolumn{2}{c}{} \\
		 
		\textsc{Version Control Systems}  & \textsc{Git}, \textsc{SVN}, Own versioning used on cloud storage for other things: \textsc{Google Drive}, \textsc{Confluence Wiki}, \textsc{iCloud} \\    
		 
	\end{tabular}
	
\end{absolutelynopagebreak}
	
	%\newpage
	%\hypertarget{gmat}{\textsc{Gmat}\setmainfont{LMRoman10 Regular}\textregistered\setmainfont[SmallCapsFont=Fontin-SmallCaps]{Fontin-Regular}}
	
	%\XeTeXpdffile ''GMAT.pdf'' page 1 scaled 800

\end{document}
